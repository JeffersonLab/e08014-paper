
The new JLab experiment, E08-014, was carried out in Hall A in 2011~\cite{e08014_pr} and focused on the measurements of inclusive cross sections at large $x$ with high precision. An high intensity electron beam with a constant energy of $3.356~\mathrm{GeV}$ was employed to the hall and struck on fixed targets, and the scattered electrons were simultaneously detected by two identical High-Resolution Spectrometers (HRSs)~\cite{halla_nim}. Three 20~cm long cryogenic targets, the $\mathrm{^{2}D}$ liquid, the $\mathrm{^{3}He}$ gas and the $\mathrm{^{4}He}$ gas, were used, in addition to thin foils of $\mathrm{^{12}C}$, $\mathrm{^{40}Ca}$ and $\mathrm{^{48}Ca}$. Each HRS consists of a pair of vertical drift chambers (VDCs) for particle tracking, two scintillator planes for triggering and timing measurements, and a gas \v{C}erenkov counter and two layers of lead-glass calorimeters for particle identification. The spectrometers were positioned at $\theta_{0}=21^\circ$, $23^\circ$, $25^\circ$, and $28^\circ$ with total of 9 different central momentum settings, which cover the $\mathrm{Q^{2}}$ range from 1.1~$\mathrm{(GeV/c)^{2}}$ upto 2.5~$\mathrm{(GeV/c)^{2}}$. The detailed description of the experiment and the data analysis can be found in Ref.~\cite{zye_thesis}.

% PID cuts and efficiencies
 The HRS detectors had very high electron detection capability and the event rate of this experiment was low since the cross section drops exponentially away from the QE peak. Events with only one-track from the VDCs' tracking reconstruction were kept for analysis, while the zero-track and multi-track events were less than 1\%. The efficiencies of the detectors were carefully evaluated and turned out to be close to 100\% even with the highest event rates. No further correction was applied. The electrons were identified by applying combination cuts on the calibrated signals from both the \v{C}erenkov detector and the calorimeters. The cuts were able to keep above 99\% electrons while the pion to electron ratio was estimated to be better than $\mathrm{10^{-4}}$ level thanks to the low pion production rate under these kinematic regions. The overall dead-time of our data acquisition system (DAQ) was evaluated and corrected for each run.

% Acceptance
The scattered electron's outgoing momentum, in-plan and out-of-plan angles and its vertex position on the target can be reconstructed with the optics matrices of the HRSs using the tracking information from the VDCs as inputs. The optics matrices have been well calibrated by many previous Hall A experiments and it were also optimized with the new calibration data taken during this experiment.s The uncertainty from the optics reconstruction is believed to be better than $99\%$~\cite{halla_nim}. To reduce the edge effects due to the spectrometers' geometries, only the central acceptance regions were chose by cutting on these reconstructed quantities. A Monte Carlo (MC) simulation of the HRSs~\cite{zye_thesis} was employed to evaluate and correct for the residual acceptance effect. 

%targets
For the cryogenic targets, we removed the contaminated events from electrons scattering off the end-cups of the target cells by applying a cut on the reconstructed vertex position of the scattered electron on the target. A dummy target of two thin aluminum foils with 20~cm apart was used to evaluate the level of residual contamination after the cut. With the precise optics reconstruction, a cut of $\pm 7~cm$ at the center part of the target is able to remove $>99.9\%$ of the events from target end-cups.

One of the largest sources of systematic uncertainty come from the non-uniform target densities of $\mathrm{^{2}D}$, $\mathrm{^{3}He}$ and $\mathrm{^{4}He}$ due to the not well distributed coolant flow and the high beam current from $40~mu\mathrm{A}$ up to $120~mu\mathrm{A}$.  We took the boiling study data on these target with varying beam currents, and extrapolated the density profiles when the beam was off and normalized the distribution to the values obtained during the target installation. For safety, we assigned 5\% uncertainty of the target density for each cryogenic target. 

(FIX-HERE: Discuss more about the systematic uncertainties from different sources and make a table).

The cross sections were extracted by binning the data in $x$. A yield ratio method was developed to apply all necessary corrections only on the MC data until the MC yield converges to the experimental yield in the same x bin. The experimental yield for the $ith$ x bin is defined as:
\begin{equation}
Y^{i}_{EX} = \frac{N^{i}_{EX}}{N_{e}},
	\end{equation}
	where $N^{i}_{EX}$ is the number of scattered electrons within the bin after all event selections and acceptance cuts, and $N_{e}$ is the total number of incoming electrons hit on the target. The MC yield can be written as:
	\begin{equation}
	Y^{i}_{MC} = \epsilon_{eff} \cdot \eta_{tg}\cdot \frac{\Delta E'_{MC}\Delta \Omega_{MC}} {N_{MC}^{gen}}\cdot\sum_{j\in i}\sigma^{rad}_{model}(E_{j},E'_{j},\theta_{j}) .
	\label{eqymc}
	\end{equation}
	where $\epsilon_{eff}$ is the product of efficiencies from varied sources. $\eta_{tg}$ denotes the areal density of the scattering centers calculated from the target thickness, $\eta_{tg}=\rho\cdot l \cdot N_{A}/A$ where $N_{A}$ is the Avogadro's number. The target density has been corrected with the boiling effect. $\Delta E'_{MC}$ and $\Delta\Omega_{MC}$ are the momentum and solid angle overages of the HRS in the simulation which were chose to be slighly larger than the actuall ranges, and $N_{MC}^{gen}$ is the total number of generated MC events within the ranges. The terms in $\sum_{j\in i}$ represent that within the $ith$ x-bin, the $jth$ electron is first weighed by the radiated cross section calculated with the incoming beam energy $E_{j}$, outgoing momentum $E'_{j}$ and angle $\theta_{j}$ and then summurized. A cross section model was developed based on the $F(y)$ scaling and the peaking-approximation method was used to calculate the radiation effect~\cite{zye_thesis}. 

	The experimental cross section for the $ith$ bin is then given by:
	\begin{equation}
	\sigma_{EX}(E, E'_{i},\theta_{0}) = \frac{Y^{i}_{EX}}{Y^{i}_{MC}}\cdot\sigma_{model}(E, E'_{i},\theta_{0}),
	\end{equation}
	where $E$ is the beam energy fixed at 3.356 GeV, $\theta_{0}$ is the central scattering angle, $E'_{i}$, the scattered energy, is calculated based on $x_{i}$, and $\sigma_{model}(E, E'_{i},\theta_{0})$ is the cross section of the bin calculated from the model with the radiation effect corrected.  In this method, the bin-centering correction was automatically applied for choosing the center of the x-bin. The cross sections of different targets were extracted with exactly the same bins and the same acceptance cuts. Their statistical and systematic errors were individually calculated before taking the cross section ratio.


