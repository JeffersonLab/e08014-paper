% conclusions


We have performed high-statistics measurements of the $^4$He/$^3$He and $^{12}$C/$^3$He cross section
ratios, confirming the results of the low-statistics measurements from Hall C~\cite{fomin2012} and showing
a clear disagreement with the CLAS data~\cite{PhysRevLett.96.082501}. This supports the idea that the CLAS
data were limited at large $x$ by bin-migration effects due to the spectrometer's modest momentum
resolution~\cite{Higinbotham:2014xna}. We do not observe the plateau predicted by the naive SRC model, but
explain why the prediction for the inclusive ratios in the 3N-SRC regime are not as robust as those for
2N-SRC. While we do not observe the predicted plateau, this does not demonstrate that 3N-SRCs are
unimportant in this region. Even if the cross section is dominated by 3N-SRCs, the inclusive scattering
ratios may not show a plateau due to the motion of the 3N-SRCs.

While the A/$^3$He ratios do not provide a direct signature of 3N-SRCs, it should still be possible to use
inclusive scattering to look for contributions of 3N configurations in nuclei. The biggest obstacle appears
to be the limited region in $x$ where the correction for the motion of any 3N-SRCs in heavy nuclei is small.
This problem can be avoided if one compares the $^3$He scattering at large $x$ with a model of the
contributions of moving 2N-SRCs in $^3$He. The contribution of 3N-SRCs would appear as an increase in the
cross section relative to what is expected when modeling scattering from $^3$He in terms of single-particle
strength and 2N-SRC contributions, including precise, quantitative corrections for the motion of the
2N-SRCs. However, because this is a comparison to theory, rather than a comparison of SRCs within two
nuclei, one can no longer rely on final-state interactions canceling in the comparison, and these effects
would have to be modeled.

It will be important for such comparisons to be performed over a range of $Q^2$, making the data to be taken
at Jefferson Lab after the energy upgrade important for such studies~\cite{e1206105}. In addition,
comparisons of scattering from $^3$He and $^3$H at large $x$~\cite{e1211112} allow for comparison of the
isospin structure in the high-momentum components of the $^3$H and $^3$He nucleon momentum distributions. If
only 2N-SRCs contribute at large momenta, then the observed n-p pair dominance will yield nearly identical
cross sections for the $x>2$ region as well, while contributions from 3N-SRCs need not be isospin independent.


