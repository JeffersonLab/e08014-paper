The results reported here are from JLab experiment E08-014~\cite{e08014_pr}, carried out in Hall A and focused on precise measurements of the
$x$ and $Q^2$ dependence of the A/$^3$He cross section ratio at large $x$. An electron beam with an energy of $3.356$~GeV and currents ranging
from XX to YY $\mu$A impinged on nuclear targets and the scattered electrons were detected in two nearly identical High-Resolution
Spectrometers (HRSs)~\cite{halla_nim}. Three 20~cm cryogenic targets were used, liquid $^2$H and gaseous $^3$He and $^4$He, along with thin foils
of $\mathrm{^{12}C}$, $\mathrm{^{40}Ca}$ and $\mathrm{^{48}Ca}$. Each HRS consists of a pair of vertical drift chambers (VDCs) for particle
tracking, two scintillator planes for triggering and timing measurements, and a gas \v{C}erenkov counter and two layers of lead-glass calorimeters
for particle identification~\cite{halla_nim}. Scattering was measured at $\theta_{0}=21^\circ$, $23^\circ$, $25^\circ$, and $28^\circ$,
cover a $Q^2$ range of 1.3--2.2~GeV$^2$ A detailed description of the experiment and data analysis can be found in Ref.~\cite{zye_thesis}.


% PID cuts and efficiencies

The analysis keeps events where a single track is identified, with very small corrections for multi-track events as the event rates are modest
for the large-$x$ kinematics. The trigger and tracking inefficiencies are extremely small and applied as a correction to the measured yield.
Electrons are identified by applying cuts on the signals from both the \v{C}erenkov detector and the calorimeters. The cuts yield $>99$\% electron
efficiency with negligible pion contamination. The overall dead-time of the data acquisition system (DAQ) was evaluated run-by-run and this
correction was applied to the measured yield.

% Acceptance

The scattered electron momentum, in-plane and out-of-plane angles, and vertex position at the target can be reconstructed with the optics matrices
of the HRSs using the tracking information from the VDCs as inputs. The optics matrices have been well calibrated by previous experiments and
were also optimized with the new calibration data taken during this experiment. To reduce the edge effects due to the spectrometers' geometries,
only the central acceptance regions were chose by cutting on these reconstructed quantities. A Monte Carlo (MC) simulation of the
HRSs~\cite{zye_thesis} was used to correct for the residual acceptance effect.
%
\textit{JRA: Probably need to at least mention modified tune/optics of right arm.  We'll probably refer to it in context of check of left arm
data, even if we don't use it. I assume that the plan is to show left arm only, as we don't need optimal statistics in this case.}.

%targets

For the cryogenic targets, we exclude events which come from scattering in the cell walls by applying a cut on the reconstructed vertex position of
the scattered electron on the target. A dummy target of two thin aluminum foils with 20~cm apart was used to evaluate the level of residual
endcap contribution after the cut. We apply a cut $\pm 7$~cm around the center of the target target, removing $>99.9\%$ of the events from target
endcap scattering. One of the largest sources of systematic uncertainty comes from target density reduction due to heating of the $^2$H, $^3$He,
and $^4$He targets in the high-current electron beam. We made dedicated measurements varying beam currents and used the variation of the yield to
measure the current dependence of the target density. This was used to determine the effective target length at the current of the measurement.
We assigned a conservative uncertainty of 5\% on the target density for each cryogenic target.

(FIX-HERE: Discuss more about the dominant systematic uncertainties).

The measured events, corrected for inefficiencies and normalized to the integrated luminosity, were binned in $x$ and compared to the simulated
yield. The simulation uses a $y$-scaling cross section model with radiative corrections applied using the peaking approximation~\cite{zye_thesis}.
\textit{JRA: Probably want reference to paper with RC formalism used}. For each $x$ bin, the ratio of experimental to Monte Carlo yield is applied
as a correction to the cross section model at that $x$ value to extract the cross section.

\textit{JRA: How large are Coulomb corrections if we (a) exclude or (b) include Calcium?}

%The experimental cross section for the $ith$ bin is then given by:
%	\begin{equation}
%	\sigma_{EX}(E, E'_{i},\theta_{0}) = \frac{Y^{i}_{EX}}{Y^{i}_{MC}}\cdot\sigma_{model}(E, E'_{i},\theta_{0}),
%	\end{equation}
%where $E$ is the beam energy fixed at 3.356 GeV, $\theta_{0}$ is the central scattering angle, $E'_{i}$, the scattered energy, is calculated based on $x_{i}$, and $\sigma_{model}(E, E'_{i},\theta_{0})$ is the cross section of the bin calculated from the model with the radiation effect corrected.  In this method, the bin-centering correction was automatically applied for choosing the center of the x-bin. The cross sections of different targets were extracted with exactly the same bins and the same acceptance cuts. Their statistical and systematic errors were individually calculated before taking the cross section ratio.
%
%{\bf Talk about isoscalar correction not being apply because of the np dominance (JRA: Not till ratios at the earliest). Coulomb correction.}
