% ****** Start of file apssamp.tex ******
%
%   This file is part of the APS files in the REVTeX 4.1 distribution.
%   Version 4.1r of REVTeX, August 2010
%
%   Copyright (c) 2009, 2010 The American Physical Society.
%
%   See the REVTeX 4 README file for restrictions and more information.
%
% TeX'ing this file requires that you have AMS-LaTeX 2.0 installed
% as well as the rest of the prerequisites for REVTeX 4.1
%
% See the REVTeX 4 README file
% It also requires running BibTeX. The commands are as follows:
%
%  1)  latex apssamp.tex
%  2)  bibtex apssamp
%  3)  latex apssamp.tex
%  4)  latex apssamp.tex
%
\documentclass[
reprint, superscriptaddress,
	%groupedaddress,
	%unsortedaddress,
	%runinaddress,
	%frontmatterverbose, 
	%preprint,
	showpacs,
	%preprintnumbers,
	%nofootinbib,
	%nobibnotes,
	%bibnotes,
	amsmath,amssymb,
	aps,
	prl,
	%pra,
	%prb,
	%rmp,
	%prstab,
	%prstper,
	floatfix,
	]{revtex4-1}

	\usepackage{graphicx} % Include figure files
	\usepackage{dcolumn}  % Align table columns on decimal point
	\usepackage{bm}       % bold math
	\usepackage{hyperref} % add hypertext capabilities
	\usepackage[mathlines]{lineno} % Enable numbering of text and display math
	\linenumbersep1mm
	\linenumbers             % Commence numbering lines

	\def\gtorder{\mathrel{\raise.3ex\hbox{$>$}\mkern-14mu
	 \lower0.6ex\hbox{$\sim$}}}
	\def\ltorder{\mathrel{\raise.3ex\hbox{$<$}\mkern-14mu
	 \lower0.6ex\hbox{$\sim$}}}

	\begin{document}

%	\title{Measurement of short-range correlations in nuclei at $x>1$ in inclusive quasi-elastic electron scattering}
	\title{Search for three-nucleon short-range correlations in nuclei}

		
\newcommand*{\JLAB}{Thomas Jefferson National Accelerator Facility, Newport News, VA 23606}
\newcommand*{\TLV}{Tel Aviv University, Tel Aviv 69978, Israel}
\newcommand*{\MIT}{Massachusetts Institute of Technology, Cambridge, MA 02139}
\newcommand*{\KENT}{Kent State University, Kent, OH 44242}
\newcommand*{\DOMINION}{Old Dominion University, Norfolk, VA 23529}
\newcommand*{\CALIF}{California State University, Los Angeles, Los Angeles, CA 90032}
\newcommand*{\Hampton}{Hampton University, Hampton, VA 23668}
\newcommand*{\PENNSYLVANIA}{Pennsylvania State University, State College, PA 16801}
\newcommand*{\Paris}{Institut de Physique Nucl\'{e}aire (UMR 8608), CNRS/IN2P3 - Universit\'e Paris-Sud, F-91406 Orsay Cedex, France}
\newcommand*{\Syracuse}{Syracuse University, Syracuse, NY 13244}
\newcommand*{\Kentucky}{University of Kentucky, Lexington, KY 40506}
\newcommand*{\William}{College of William and Mary, Williamsburg, VA 23187}
\newcommand*{\Virginia}{University of Virginia, Charlottesville, VA 22904}
\newcommand*{\Halifax}{Saint Mary's University, Halifax, Nova Scotia, Canada}
\newcommand*{\Glasgow}{University of Glasgow, Glasgow G12 8QQ, Scotland, United Kingdom}
\newcommand*{\Temple}{Temple University, Philadelphia, PA 19122}
\newcommand*{\Argonne}{Physics Division, Argonne National Laboratory, Argonne, IL 60439}
\newcommand*{\China}{China Institute of Atomic Energy, Beijing, China}
\newcommand*{\NRCN}{Nuclear Research Center Negev, Beer-Sheva, Israel}
\newcommand*{\Catania}{Universita di Catania, Catania, Italy}
\newcommand*{\Dequense}{Duquesne University, Pittsburgh, PA 15282}
\newcommand*{\Pittsburgh}{Carnegie Mellon University, Pittsburgh, PA 15213}
\newcommand*{\LongwoodUniv}{Longwood University, Farmville, VA 23909}
\newcommand*{\Florida}{Florida International University, Miami, FL 33199}
\newcommand*{\Tallahassee}{Florida State University, Tallahassee, FL 32306}
\newcommand*{\INFN}{INFN, Sezione Sanit\`{a} and Istituto Superiore di Sanit\`{a}, 00161 Rome, Italy}
\newcommand*{\INFNBari}{INFN, Sezione di Bari and University of Bari, I-70126 Bari, Italy}
\newcommand*{\Ohio}{Ohio University, Athens, OH 45701}
\newcommand*{\Tennessee}{University of Tennessee, Knoxville, TN 37996}
\newcommand*{\Kharkov}{Kharkov Institute of Physics and Technology, Kharkov 61108, Ukraine}
\newcommand*{\LOSALAMOS}{Los Alamos National Laboratory, Los Alamos, NM 87545}
\newcommand*{\Duke}{Duke University, Durham, NC 27708}
\newcommand*{\Texas}{University of Texas, Houston, TX 77030}
\newcommand*{\Seoul}{Seoul National University, Seoul, Korea}
\newcommand*{\Indiana}{Indiana University, Bloomington, IN 47405}
\newcommand*{\Hampshire}{University of New Hampshire, Durham, NH 03824}
\newcommand*{\Blacksburg}{Virginia Polytechnic Inst. and State Univ., Blacksburg, VA 24061}
\newcommand*{\France}{Universit\'e Blaise Pascal/IN2P3, F-63177 Aubi\`ere, France}
\newcommand*{\Mississippi}{Mississippi State University, Mississippi State, MS 39762}
\newcommand*{\Austin}{The University of Texas at Austin, Austin, Texas 78712}
\newcommand*{\Norfolk}{Norfolk State University, Norfolk, VA 23504}
\newcommand*{\Lanzhou}{Lanzhou University, Lanzhou, China}
\newcommand*{\Hebrew}{Racah Institute of Physics, Hebrew University of Jerusalem, Jerusalem, Israel}
\newcommand*{\Rutgers}{Rutgers, The State University of New Jersey, Piscataway, NJ 08855}
\newcommand*{\Yerevan}{Yerevan Physics Institute, Yerevan 375036, Armenia}
\newcommand*{\Ljubljana}{University of Ljubljana, Ljubljana, Slovenia}
\newcommand*{\Michigan}{Northern Michigan University, Marquette, MI 49855}
\newcommand*{\Hefei}{University of Science and Technology, Hefei, China}	
\newcommand*{\Jozef}{Jozef Stefan Institute, Ljubljana, Slovenia}
\newcommand*{\Ecole}{CEA Saclay, F-91191 Gif-sur-Yvette, France}
\newcommand*{\Massachusetts}{University of Massachusetts, Amherst, MA 01006}

\author{Z. Ye}
\affiliation{\Argonne}
\affiliation{\Virginia}
\affiliation{\Duke}
\author{P. Solvignon}
\affiliation{\Hampshire}
\affiliation{\JLAB}
\author{D. Nguyen}
\affiliation{\Virginia}
\author{P. Aguilera}
\affiliation{\Paris}
\author{Z. Ahmed}
\affiliation{\Syracuse}
\author{H. Albataineh}
\affiliation{\DOMINION}
\author{K. Allada}
\affiliation{\JLAB}
\author{B. Anderson}
\affiliation{\KENT}
\author{D. Anez}
\affiliation{\Halifax}	
\author{K. Aniol}
\affiliation{\CALIF}
\author{J. Annand}
\affiliation{\Glasgow}
\author{J. Arrington}
\affiliation{\Argonne}
\author{T. Averett}
\affiliation{\William}
\author{H. Baghdasaryan}
\affiliation{\Virginia}
\author{X. Bai}
\affiliation{\China}
\author{A. Beck}
\affiliation{\NRCN}	
\author{S. Beck}
\affiliation{\NRCN}	
\author{V. Bellini}
\affiliation{\Catania}
\author{F. Benmokhtar}
\affiliation{\Dequense}
\author{A. Camsonne}
\affiliation{\JLAB}
\author{C. Chen}
\affiliation{\Hampton}
\author{J.-P. Chen}
\affiliation{\JLAB}
\author{K. Chirapatpimol}
\affiliation{\Virginia}
\author{E. Cisbani}
\affiliation{\INFN}
\author{M.~M. Dalton}
\affiliation{\Virginia}
\affiliation{\JLAB}
\author{A. Daniel}
\affiliation{\Ohio}
\author{D. Day}
\affiliation{\Virginia}
\author{W. Deconinck}
\affiliation{\MIT}
\author{M. Defurne}
\affiliation{\Ecole}	
\author{D. Flay}
\affiliation{\Temple}
\author{N. Fomin}
\affiliation{\Tennessee}
\author{M. Friend}
\affiliation{\Pittsburgh}
\author{S. Frullani}
\affiliation{\INFN}
\author{E. Fuchey}
\affiliation{\Temple}
\author{F. Garibaldi}
\affiliation{\INFN}
\author{D. Gaskell}
\affiliation{\JLAB}
\author{S. Gilad}
\affiliation{\MIT}
\author{R. Gilman}
\affiliation{\Rutgers}
\author{S. Glamazdin}
\affiliation{\Kharkov}
\author{C. Gu}
\affiliation{\Virginia}
\author{P. Gu\`eye}
\affiliation{\Hampton}
\author{C. Hanretty}
\affiliation{\Virginia}
\author{J.-O. Hansen}
\affiliation{\JLAB}
\author{M. Hashemi Shabestari}
\affiliation{\Virginia}
\author{O. Hen}
\affiliation{\TLV}
\author{D.~W. Higinbotham}
\affiliation{\JLAB}
\author{M. Huang}
\affiliation{\Duke}
\author{S. Iqbal}
\affiliation{\CALIF}
\author{G. Jin}
\affiliation{\Virginia}
\author{N. Kalantarians}
\affiliation{\Virginia}
\author{H. Kang}
\affiliation{\Seoul}
\author{A. Kelleher}
\affiliation{\MIT}
\author{I. Korover}
\affiliation{\TLV}
\author{J. LeRose}
\affiliation{\JLAB}
\author{J. Leckey}
\affiliation{\Indiana}	
\author{R. Lindgren}
\affiliation{\Virginia}
\author{E. Long}
\affiliation{\KENT}
\author{J. Mammei}
\affiliation{\Blacksburg}
\author{D. J. Margaziotis}
\affiliation{\CALIF}
\author{P. Markowitz}
\affiliation{\Florida}
\author{D. Meekins}
\affiliation{\JLAB}
\author{Z. Meziani}
\affiliation{\Temple}
\author{R. Michaels}
\affiliation{\JLAB}
\author{M. Mihovilovic}
\affiliation{\Jozef}
\author{N. Muangma}
\affiliation{\MIT}
\author{C. Munoz Camacho}
\affiliation{\France}
\author{B. Norum}
\affiliation{\Virginia}
\author{Nuruzzaman}
\affiliation{\Mississippi}
\author{K. Pan}
\affiliation{\MIT}
\author{S. Phillips}
\affiliation{\Hampshire}
\author{E. Piasetzky}
\affiliation{\TLV}
\author{I. Pomerantz}
\affiliation{\TLV}
\affiliation{\Austin}
\author{M. Posik}
\affiliation{\Temple}
\author{V. Punjabi}
\affiliation{\Norfolk}	
\author{X. Qian}
\affiliation{\Duke}	
\author{Y. Qiang}
\affiliation{\JLAB}
\author{X. Qiu}
\affiliation{\Lanzhou}
\author{P.~E. Reimer}
\affiliation{\Argonne}
\author{A. Rakhman}
\affiliation{\Syracuse}
\author{S. Riordan}
\affiliation{\Virginia}
\affiliation{\Massachusetts}
\author{G. Ron}
\affiliation{\Hebrew}
\author{O. Rondon-Aramayo}
\affiliation{\Virginia}
\author{A. Saha}
\thanks{deceased}
\affiliation{\JLAB}
\author{L. Selvy}
\affiliation{\KENT}
\author{A. Shahinyan}
\affiliation{\Yerevan}
\author{R. Shneor}
\affiliation{\TLV}
\author{S. Sirca}
\affiliation{\Ljubljana}
\author{K. Slifer}
\affiliation{\Hampshire}
\author{N. Sparveris}
\affiliation{\Temple}	
\author{R. Subedi}
\affiliation{\Virginia}
\author{V. Sulkosky}
\affiliation{\MIT}
\author{D. Wang}
\affiliation{\Virginia}
\author{J.~W. Watson}
\affiliation{\KENT}
\author{L.~B. Weinstein}
\affiliation{\DOMINION}
\author{B. Wojtsekhowski}
\affiliation{\JLAB}
\author{S.~A. Wood}
\affiliation{\JLAB}
\author{I. Yaron}
\affiliation{\TLV}
\author{X. Zhan}
\affiliation{\Argonne}
\author{J. Zhang}
\affiliation{\JLAB}
\author{Y.~W. Zhang}
\affiliation{\Rutgers}
\author{B. Zhao}
\affiliation{\William}
\author{X. Zheng}
\affiliation{\Virginia}
\author{P. Zhu}
\affiliation{\Hefei}
\author{R. Zielinski}
\affiliation{\Hampshire}	

\collaboration{The Jefferson Lab Hall A Collaboration}



		\date{\today}

		%\begin{linenomath}
		\begin{abstract}

We present new data probing short-range correlations (SRC) in nuclei through the measurement of electron scattering off high-momentum nucleons in
light nuclei. The inclusive cross section ratios of $^4$He to $^3$He and $^{12}$C to $^3$He are observed to be both $Q^2$ and $x$ independent
for $1.5 < x <2$, confirming the previously observed dominance of two-nucleon short-range corrections. We also examine the $Q^2$ dependence for
$x > 2$ where previous data suggested that scattering from three-nucleon correlations might dominate the cross section.

%We find that the cross section ratios do not agree with an earlier measurement which suggested a dominance of 3N-SRCs, and do not exhibit the
%behavior expected in a naive model of 3N-SRCs. The deviations from this simple picture may simply be due to failure of the approximations made in the
%simple model, and data at larger $Q^2$ may offer a more definitive answer on the question of 3N-SRCs.


%We also present the first measurement of the $\mathrm{^{48}Ca}$ to $\mathrm{^{40}Ca}$ cross section ratio and the result indicates the universality of two-nucleon corrections for high-momentum nucleons in the isotopes.
		\end{abstract}
		%\end{linenomath}

		\pacs{25.10.1s, 25.30.Fj}
		\maketitle

		%introduction
		
% review

Understanding the complex structure of nuclei remains one of the major tasks in nuclear physics, and several questions remain about the
high-momentum components of the nuclear wavefunction. This is an important component of nuclear structure that goes beyond the shell model
description. Mean field calculations~\cite{DeForest1983} do not include these high-momentum components, and so significantly overpredict the cross
section for proton knock-out reactions with proton momenta below the Fermi momenta~\cite{VanDerSteenhoven1988547, Lapikas1993297, Kelly:1996hd}

%Without the inclusion of these high-momentum components, the mean-field calculation using the distorted wave impulse
%approximation~\cite{DeForest1983}  overestimates the nuclear strength which had been observed by many proton knock-out
%experiments~\cite{VanDerSteenhoven1988547, Lapikas1993297, Kelly:1996hd}.

In the dense and energetic environment of the nucleus, nucleons have a significant probability of interacting at distance near or below 1~fm, even
in light nuclei~\cite{carlson14,lu13}. Protons and neutrons interacting through the strong, short-distance components of the NN potential yield
pairs of nucleons with large momenta, well above the typical scale of the Fermi momentum associated with the shell model picture of nuclei. These
pairs of high-momentum nucleons, the so-called short-range correlations (SRCs), are the dominant source of the high-momentum part of the nuclear
momentum distribution~\cite{SLAC_Measurement_PRC.48.2451, src_john}. Thus, the nuclear momentum distribution has two distinct regions, driven by
very different physics. For momenta below the Fermi momentum, $k_F \approx 300$~MeV/c, we have collective, shell-model behavior which varies
rapidly with A. For momenta above $k_F$, two-body physics dominates and there is a universal structure for all nuclei, driven by the details
of the two-body NN interaction~\cite{RevModPhys.80.189, PhysRevC.53.1689, wiringa14}.

Because the momentum distribution of the nucleus is not an experimental observable, one cannot simply measure this for all nuclei and compare the
high-momentum components. One can, however, test the idea of a universal structure to the high-momentum components by comparing quasi-elastic
scattering from different nuclei at kinematics which require that the struck nucleon have a large initial momentum~\cite{RevModPhys.80.189}.
During the scattering, the electron transfers energy, $\nu$, and momentum $\vec{q}$ to the struck nucleon by exchanging a virtual photon with four
momentum transfer $q^2 = - Q^{2} = \nu^{2}-|\vec{q}|^{2}$. It is useful in this case to define the kinematic variable $x = Q^2/(2M_p\nu)$, where
$M_p$ is the mass of the proton.
%
%This is identical to the Bjorken variable $x$, which correspond to the momentum fraction of the struck quark in deeply-inelastic scattering at
%higher energies, but in the present case, we are looking at coherent scattering from a single nucleon.
%
Elastic scattering from a stationary proton corresponds to $x=1$, while $x>1$ corresponds to high momentum transfer but low energy transfer.
This requires that the virtual photon be absorbed by a proton whose initial momentum is opposite to the momentum of the virtual photon, so that
the large transferred momentum changes the direction of the proton's momentum while minimizing the magnitude of the final momentum and thus the
proton's kinetic energy. Scattering at $x>1$ must involve more than one nucleon as the initial momentum of the struck nucleon must be balanced
by one or more nucleons in the nucleus. The kinematic limit for scattering from a nucleus is $x = M_A/M_p \approx A$, which for scattering
from the deuteron corresponds to $x \approx 2$; scattering at $x>2$ thus involves the participation of at least three nucleons.

Based on these kinematic arguments, one can use $x$ to determine the minimum number of nucleons involved in the interaction. Values of $x$ 
slightly greater than unity requires only a small momentum which can come from either the single-particle contributions or from high-momentum
nucleons associated with SRCs. As $x$ increases, larger momenta are required and for sufficiently large $x$ scattering from nucleons below
the Fermi momentum is forbidden, isolating scattering from SRCs~\cite{RevModPhys.80.189, PhysRevC.53.1689}. Previous measurements at SLAC
and Jefferson Lab revealed a universal form to the high-momentum distributions of the struck nucleons~\cite{SLAC_Measurement_PRC.48.2451,
egiyan2003, PhysRevLett.96.082501, fomin2012, src_john}. In these experiments, the ratio of scattering from a heavy nucleus to the deuteron was
shown to scale, i.e. be independent of $x$ and $Q^2$, for $x \gtorder 1.5$ and $Q^2 \gtorder 1.5$~GeV$^2$, corresponding to scattering from
nucleons with momenta above 300 MeV/c. Other measurements have demonstrated that these high-momentum components are dominated by high-momentum n-p
pairs~\cite{Subedi13062008, korover2014, hen14_science}, meaning that the high-momentum components in all nuclei have a deuteron-like structure.

Taking ratios of heavier nuclei to $^3$He allows a similar examination of the target ratios for $x>2$, where one might expect to see a universal
signature of three-nucleon SRCs - configurations of three nucleons which have large relative but small total momenta. These configurations could
come from either three-nucleon forces or successive hard two-nucleon interactions. The first such measurement~\cite{PhysRevLett.96.082501}
observed ratios which were independent of $x$ above $x=2.25$, and this was taken as an indication that the 3N-SRCs dominated in this region and
extracted the relative contribution of the 3N-SRCs in heavy nuclei compared to $^3$He. However the ratios were measured at relatively low $Q^2$
and the $Q^2$ dependence was not measured. A later experiment~\cite{fomin2012} at higher $Q^2$ yielded $^4$He/$^3$He ratios that were
significantly larger than those from~\cite{PhysRevLett.96.082501}. Thus the question of whether 3N-SRC contributions have been cleanly identified
and observed to dominate at some very large momentum scale is as yet unanswered.

% JRA: Some of this may be better/more concise than bit above.
%
%The first evidence of SRC in inclusive scattering was revealed by the SLAC data~\cite{SLAC_Measurement_PRC.48.2451} with $a_2(A)$ exhibiting a
%plateau between $x\sim1.5$ and $x\sim2$, where 2N-SRC is expected to dominate. A recent measurement from the CLAS data in Hall B at
%JLab~\cite{PhysRevLett.96.082501} also reported the 2N-SRC plateau in the $a_3(A)$ distribution. The latest measurement from the E02-019
%experiment in Hall C at JLab with better precision and a wider range of nuclei, and both $a_2(A)$ and $a_3(A)$ show clear 2N-SRC
%plateau~\cite{fomin2012}. In the $x>2$ region, while the CLAS data claimed a second plateau at $x>2$ in the
%$\sigma_{^{4}He}/\sigma_{^{3}He}$ ratio, E02-019 sees, despite the large error bars, clearly a rise in the $a_3(A)$ distribution. It should be
%noted that both experiments reported data at very different $Q^{2}$ and the kinematical requirement in performing a clean measurement of 3N-SRC is
%not yet well understood.



%The interaction between the virtual photon and the nucleon provides an unique probe to study the nucleon's initial state, e.g., the momentum distribution which is correlated to the scaling function~\cite{West1975263,day_arns, PhysRevC.41.R2474, Boffi19931,RevModPhys.80.189}:
%\begin{equation}
%F(y) = 2\pi\int_{|y|}^{\infty}n(p_{0})\cdot p_{0}dp_{0},
%	\label{fy_mom_eq}
%	\end{equation} 
%	where $n(p_{0})$ is the momentum distribution of the nucleon with the initial momentum $p_{0}$. $y$ is the solution of $M_{A}+\nu = \sqrt{M^{2}+|\vec{q}|^{2}+y^{2}+2y|\vec{q}|}+\sqrt{M_{A-1}^{2}+y^{2}}$ where $M$, $M_{A}$ and $M_{A}$ are the masses of the nucleon, target nucleus A and the (A-1) recoil system, respectively. $F(y)$ can be directly extracted from the experimental QE inclusive cross section:
%	\begin{equation}
%	F(y)=\frac{d^{3}\sigma_{EX}}{dE' d\Omega } \frac{1}{Z\sigma_{p}+N\sigma_{n}} \frac{q}{\sqrt{M^{2}+(y+q)^{2}}},
%	\label{fy_scaling_eq2}
%	\end{equation}
%	where $\sigma_{p}$ and $\sigma_{n}$ are the electron-proton and electron-neutron cross section, respectively.


%Compared with the electron elastic scattering process which is well peaked at $x=Q^{2}/2Mv=1$ (M is the proton mass), the QE process yields a much
%broader peaks at $x=1$ due to the Fermi motion of the nucleon inside the nucleus. By measuring the inclusive QE cross section at $x>1$ with
%$Q^{2}>1~GeV^{2}$, one can carefully map out the SRC in different nuclei by taking the ratio of the cross section per nucleon of the heavy nucleus,
%$A$, to the light nucleus, e.g. deuteron or $\mathrm{^{3}He}$:
%\begin{equation}
%a_{2}(A) = \frac{\sigma_{A}(x,Q^{2})/A}{\sigma_{D}(x,Q^{2})/2},~~~  a_3(A) = \frac{\sigma_{A}(x,Q^{2})/A}{\sigma_{^{3}He}(x,Q^{2})/3},
%\end{equation}



%%{\bf (talk about Doug and Or's Analysis, Phys.Rev.Lett. 114 (2015) 16, 169201, when we discuss our new results)}




		%experiment
		The results reported here are from JLab experiment E08-014~\cite{e08014_pr}, carried out in Hall A and focused on precise measurements of the
$x$ and $Q^2$ dependence of the A/$^3$He cross section ratio at large $x$. An electron beam with an energy of $3.356$~GeV and currents ranging
from XX to YY $\mu$A impinged on nuclear targets and the scattered electrons were detected in two nearly identical High-Resolution
Spectrometers (HRSs)~\cite{halla_nim}. Three 20~cm cryogenic targets were used, liquid $^2$H and gaseous $^3$He and $^4$He, along with thin foils
of $\mathrm{^{12}C}$, $\mathrm{^{40}Ca}$ and $\mathrm{^{48}Ca}$. Each HRS consists of a pair of vertical drift chambers (VDCs) for particle
tracking, two scintillator planes for triggering and timing measurements, and a gas \v{C}erenkov counter and two layers of lead-glass calorimeters
for particle identification~\cite{halla_nim}. Scattering was measured at $\theta_{0}=21^\circ$, $23^\circ$, $25^\circ$, and $28^\circ$,
cover a $Q^2$ range of 1.3--2.2~GeV$^2$ A detailed description of the experiment and data analysis can be found in Ref.~\cite{zye_thesis}.


% PID cuts and efficiencies

The analysis keeps events where a single track is identified, with very small corrections for multi-track events as the event rates are modest
for the large-$x$ kinematics. The trigger and tracking inefficiencies are extremely small and applied as a correction to the measured yield.
Electrons are identified by applying cuts on the signals from both the \v{C}erenkov detector and the calorimeters. The cuts yield $>99$\% electron
efficiency with negligible pion contamination. The overall dead-time of the data acquisition system (DAQ) was evaluated run-by-run and this
correction was applied to the measured yield.

% Acceptance

The scattered electron momentum, in-plane and out-of-plane angles, and vertex position at the target can be reconstructed with the optics matrices
of the HRSs using the tracking information from the VDCs as inputs. The optics matrices have been well calibrated by previous experiments and
were also optimized with the new calibration data taken during this experiment. To reduce the edge effects due to the spectrometers' geometries,
only the central acceptance regions were chose by cutting on these reconstructed quantities. A Monte Carlo (MC) simulation of the
HRSs~\cite{zye_thesis} was used to correct for the residual acceptance effect.
%
\textit{JRA: Probably need to at least mention modified tune/optics of right arm.  We'll probably refer to it in context of check of left arm
data, even if we don't use it. I assume that the plan is to show left arm only, as we don't need optimal statistics in this case.}.

%targets

For the cryogenic targets, we exclude events which come from scattering in the cell walls by applying a cut on the reconstructed vertex position of
the scattered electron on the target. A dummy target of two thin aluminum foils with 20~cm apart was used to evaluate the level of residual
endcap contribution after the cut. We apply a cut $\pm 7$~cm around the center of the target target, removing $>99.9\%$ of the events from target
endcap scattering. One of the largest sources of systematic uncertainty comes from target density reduction due to heating of the $^2$H, $^3$He,
and $^4$He targets in the high-current electron beam. We made dedicated measurements varying beam currents and used the variation of the yield to
measure the current dependence of the target density. This was used to determine the effective target length at the current of the measurement.
We assigned a conservative uncertainty of 5\% on the target density for each cryogenic target.

(FIX-HERE: Discuss more about the dominant systematic uncertainties).

The measured events, corrected for inefficiencies and normalized to the integrated luminosity, were binned in $x$ and compared to the simulated
yield. The simulation uses a $y$-scaling cross section model with radiative corrections applied using the peaking approximation~\cite{zye_thesis}.
\textit{JRA: Probably want reference to paper with RC formalism used}. For each $x$ bin, the ratio of experimental to Monte Carlo yield is applied
as a correction to the cross section model at that $x$ value to extract the cross section.

\textit{JRA: How large are Coulomb corrections if we (a) exclude or (b) include Calcium?}

%The experimental cross section for the $ith$ bin is then given by:
%	\begin{equation}
%	\sigma_{EX}(E, E'_{i},\theta_{0}) = \frac{Y^{i}_{EX}}{Y^{i}_{MC}}\cdot\sigma_{model}(E, E'_{i},\theta_{0}),
%	\end{equation}
%where $E$ is the beam energy fixed at 3.356 GeV, $\theta_{0}$ is the central scattering angle, $E'_{i}$, the scattered energy, is calculated based on $x_{i}$, and $\sigma_{model}(E, E'_{i},\theta_{0})$ is the cross section of the bin calculated from the model with the radiation effect corrected.  In this method, the bin-centering correction was automatically applied for choosing the center of the x-bin. The cross sections of different targets were extracted with exactly the same bins and the same acceptance cuts. Their statistical and systematic errors were individually calculated before taking the cross section ratio.
%
%{\bf Talk about isoscalar correction not being apply because of the np dominance (JRA: Not till ratios at the earliest). Coulomb correction.}

		%results	
		%results


                \begin{figure}[!ht]
		\begin{center}
		\includegraphics[height=9.0cm,angle=270]{./figures/XS_Comp_25_May27}
		\end{center}
		\vspace*{-5mm}
		\caption{(Color online) Cross sections of $^{3}$He, $^{4}$He and $^{12}$C at $25^{\circ}$. The uncertainties include statistical and
		systematic uncertainties. An additional normalization uncertainty of XX\% is not shown.}
		\label{xs}
		\end{figure}

The absolute cross sections for scattering from $^{3}$He, $^{4}$He and $^{12}$C at a scattering angle of $25^{\circ}$ are shown in Fig.~\ref{xs}.


                \begin{figure}[!ht]
		\begin{center}
		  \includegraphics[height=9.5cm,angle=270]{./figures/He4_He3_XS_Ratio_June30_L}
                  \includegraphics[height=9.5cm,angle=270]{./figures/C12_He3_XS_Ratio_June30_L}
%                  \includegraphics[height=9.0cm,angle=270]{./figures/C12_He4_XS_Ratio_June30_L}
		\end{center}
		\vspace*{-5mm}
		\caption{(Color online) The $^4$He/$^3$He (top) and $^{12}$C/$^3$He (bottom) cross section ratios for this
		  measurement, along with results from CLAS~\cite{PhysRevLett.96.082501} and Hall C (E02-019)~\cite{fomin2012} measurements.
                  Uncertainties include statistical and systematic uncertainties. A normalization uncertainty of XX\% (top) and YY\% (bottom)
		  is not included. \textit{JRA: It will be good to make these large, so making the y-axis labels less 'tall' will help, e.g.
		  something like $(\sigma_{4He}/4)/(\sigma_{3He}/3)$. Do we want to include our cross section model to show $Q^2$ dependence?}}
		\label{ratios}
		\end{figure}

Fig.~\ref{ratios} presents the ratio of the $^4$He and $^{12}$C cross sections to $^3$He as a function of $x$. In the 2N-SRC region, our data
are in good agreement with the data from CLAS~\cite{PhysRevLett.96.082501} and E02-019~\cite{fomin2012}, revealing a plateau between $x \approx
1.5$ and $x = 2$. At $x>2$, our ratios are significantly larger than the CLAS ratios, and in generally good agreement with the E02-019 ratios.
The disagreement between the CLAS ratios and both our results and the E02-019 data suggest a problem with the extracted ratios from
Ref.~\cite{PhysRevLett.96.082501} above $x=2$. A recent comment~\cite{Higinbotham:2014xna} suggested that the 3N-SRC plateau showed in the CLAS
data could be a result of inappropriate binning and bin-centering correction.

We observe a small but systematic $Q^2$ dependence in our data, and do not see indications of a well defined 3N-SRC plateau. Instead our ratios
show a slow rise above $x=2$, with a more rapid increase as $x \to 3$, suggesting that the simple model of 3N-SRC dominance is not valid in this
region. While this behavior does not match the prediction of the native 3N-SRC model, namely A/$^3$He ratios independent of $x$ and $Q^2$ for $x
\gtorder 2.5$, this does not provide a clear demonstration that 3N-SRCs are unimportant in this region.

For 2N-SRCs, the prediction of scaling is relatively straightforward and robust. One can predict \textit{a priori} where the plateau should be
observed since for any given $Q^2$, a value of $x$ can be selected that corresponds to a minimum nucleon momentum that is above the Fermi
momentum, thus suppressing the mean-field contributions. As one approaches $x=2$, the plateau will disappear as the deuteron cross section falls
to zero and so the A/$^2$H ratios must rise sharply to infinity. For both the data and our simple cross section model, based on a calculated
deuteron momentum distribution, this does not occur until $x \approx 1.9$, yielding a clear plateau for $1.5 < x < 1.9$.

In attempting to isolate 3N-SRC contributions, the situation is less straightforward. Both 2N and 3N SRCs yield contributions to the 
high-momentum tail. The fact that we do not see significant deviations from the 2N-SRC picture for $1.5<x<2$ suggests that the 3N-SRC
contributions are generally much smaller. Unlike the case for 2N-SRCs, where $k>k_F$ suppresses single particle strength, there is
not a clear way to define a threshold in $x$ that will sufficiently suppress
2N contributions.  Approaching the kinematic limit at $x \approx 3$, the $^3$He cross section falls to zero and the ratio must go to
infinity. However, while this occurs in a vary narrow $x$ window for the A/$^2$H ratios, the rise occurs over a larger range in $x$
in this case.

Thus, it is not clear that there will be a significant window in $x$ where one would expect to see a plateau, especially at the relatively modest
$Q^2$ values measured here. In the present experiment, we observe a small but noticeable $Q^2$ dependence, in particular for $x \gtorder 2.5$. This
is also observed in our simple $y$-scaling cross section model, and does not occur in the $^{12}$C/$^3$He ratio, indicating that it is the $x$
dependence of the falloff of the $^3$He cross section as $x \to 3$ that is varying strongly with $Q^2$. Larger $Q^2$ values may be required to
observe a $Q^2$-independent behavior of the ratios with $x$, which may allow us to isolate 3N-SRC contributions.



\textit{Figure with A/2H ratios up to $x=2$, table with $a_2$ results for $A=3$, 4, 12, 48?}


%JRA Notes:
%0) Can we look at A/D ratios vs. Q^2 for our cross section model? Impact of x-->2 in 2H? Where does A/D rise up for x-->2?
%1) x-->3 has may have larger missing energy (plus, smaller energy step from x=1 to 2 to 3, so may fall to zero over wider range.
%2) $Q^2$ dependence of $x \to 3$ ratios in general agreement with what we observe in our cross section model. Suggests that the cross
%section falloff as we approach the kinematic threshold is large


		%discussion
			% conclusions
(Add conclusion here). Also, brief discussion of future experiments (certainly $x>1$ in Hall C, probably $x<3$ in Hall A as well).

	


		\begin{acknowledgments}
We would like to acknowledge the outstanding support from the Jefferson Lab Hall A 
technical staff and the JLab target group. This work was supported by the NSF and the
DOE, Office of Science, Office of Nuclear Physics, under contract DE-AC02-06CH11357, and by DOE 
contract DE-AC05-06OR23177 under which JSA, LLC operates JLab.
		\end{acknowledgments}

		%\bibliographystyle{apsrev4-1}
		\bibliographystyle{h-physrev3.bst}
		\bibliography{e08014}

		\end{document}
