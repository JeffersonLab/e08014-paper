
% review
Understanding the complex structure of nuclei remains as one of the major tasks in nuclear physics. Short-range correlations (SRC) play an important rule to the formation of the nuclear structure but yet be well studied. The mighty repulsive core of the nucleon-nucleon (NN) interactions at short distance boosts the correlated nucleons' well above the Fermi momentum, while the nucleus remains in its ground state due to the momentum conservation. Without involving these high momentum components, the mean field calculation using the distorted wave impulse approximation~\cite{DeForest1983} underestimated the nuclear strength which had been observed by many proton knock-out experiments~\cite{VanDerSteenhoven1988547,Lapikas1993297,Kelly:1996hd}.

Previous data revealed an asymptotic form of the momentum distributions for struck nucleons orginally from light nuclei and heavy nuclei~\cite{PhysRevC.53.1689}. It showed at moderate high momentum from 300~MeV to 600~MeV, the distributions of heavy nuclei scale to the one of the deuteron. It could be easily understood if the 2N-SRC pair shares the similar features among different nuclei. One also suggested that the momentum distributions should scale to the one of $\mathrm{^{3}H}$ or $\mathrm{^{3}He}$ for the much higher momentum tail where 3N-SRC configuration dominates~\cite{src_john}.

Instead of direclty investigating the momentum distribution of the nucleus which is not an experimental abservable, one can study the SRC via inclusive electrons quasielastic (QE) scattering off nuclei~\cite{RevModPhys.80.189}. During the scattering, the electron gives up its energy via emitting a virtual photon with the four momentum transfer $Q^{2} = |\vec{q}|^{2}-\nu^{2}$, where $\vec{q}$ and $\nu$ are the momentum and energy of the virtual photon. The interaction between the virtual photon and the nucleon provides an unique probe to study the nucleon's intial state, e.g., the momentum distribution which is correlated to the scaling function~\cite{West1975263,day_arns, PhysRevC.41.R2474, Boffi19931,RevModPhys.80.189}:
\begin{equation}
F(y) = 2\pi\int_{|y|}^{\infty}n(p_{0})\cdot p_{0}dp_{0},
	\label{fy_mom_eq}
	\end{equation} 
	where $n(p_{0})$ is the momentum distribution of the nucleon with the initial momentum $p_{0}$. $y$ is the solution of $M_{A}+\nu = \sqrt{M^{2}+|\vec{q}|^{2}+y^{2}+2y|\vec{q}|}+\sqrt{M_{A-1}^{2}+y^{2}}$ where $M$, $M_{A}$ and $M_{A}$ are the masses of the nucleon, target nucleus A and the (A-1) recoil system, respectively. $F(y)$ can be directly extracted from the experimental QE inclusive cross section:
	\begin{equation}
	F(y)=\frac{d^{3}\sigma_{EX}}{dE' d\Omega } \frac{1}{Z\sigma_{p}+N\sigma_{n}} \frac{q}{\sqrt{M^{2}+(y+q)^{2}}},
	\label{fy_scaling_eq2}
	\end{equation}
	where $\sigma_{p}$ and $\sigma_{n}$ are the electron-proton and electron-neutron cross section, respectively.

	Compared with the electrons elastic scattering process which is well peaked at $x=Q^{2}/2Mv=1$ (M is the proton mass), the QE process yields a much broader peaks at $x=1$ due to the Fermi motion of the nucleon inside the nucleus. By measuring the inclusive QE cross section at $x>1$ with $Q^{2}>1~GeV^{2}$, one can carefully map out the SRC in different nuclei by taking the cross section ratio of the heavy nucleus, A, to the light nuclei, e.g. deuteron or $\mathrm{^{3}He}$:
	\begin{equation}
	a_{D}(A) = \frac{2}{A}\frac{\sigma_{A}(x,Q^{2})}{\sigma_{D}(x,Q^{2})},  a_{^{3}He}(A) = \frac{3}{A}\frac{\sigma_{A}(x,Q^{2})}{\sigma_{^{3}He}(x,Q^{2})},
	\end{equation}
	where $2/A$ or $2/A$ accounts for the possibilities of forming SRC configurations in different nuclei similar to deuteron or $\mathrm{^{3}He}$. 

	The first study of the SRC via the inclusive scattering was revealed by the SLAC data~\cite{SLAC_Measurement_PRC.48.2451} which revealed a 2N-SRC plateau on the $a_{D}(A)$ started to raise at $x\sim1.5$. A recent measurement from the CLAS data in Hall B at JLab~\cite{PhysRevLett.96.082501} reported a more clear 2N-SRC plateau on the $a_{^{3}He}(A)$ distribution. The latest measurement was performed by the E02-019 experiment from Hall C at JLab with better precision and a wider range of nuclei, and both $a_{D}(A)$ and $a_{^{3}He}(A)$ show clear 2N-SRC plateau~\cite{PhysRevLett.105.212502}. The CLAS data claimed a second plateau at $x>2$ in the $\sigma_{^{4}He}/\sigma_{^{3}He}$ ratio. However, the E02-19 result presented a different approach in this region despite the large error bars due to the lack of statistics. The discrepancy between these two measurements can not be explained at this stage. One of the facts is that both experiments ran at very different $Q^{2}$ while the kinematic requirement of performing clean measurements of 3N-SRC is not well understood yet.


