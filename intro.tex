
% review

Understanding the complex structure of nuclei remains one of the major tasks in nuclear physics, and several questions remain about the
high-momentum components of the nuclear wavefunction. This is an important component of nuclear structure that goes beyond the shell model
description. Mean field calculations~\cite{DeForest1983} do not include these high-momentum components, and so significantly overpredict the cross
section for proton knock-out reactions with proton momenta below the Fermi momenta~\cite{VanDerSteenhoven1988547, Lapikas1993297, Kelly:1996hd}

%Without the inclusion of these high-momentum components, the mean-field calculation using the distorted wave impulse
%approximation~\cite{DeForest1983}  overestimates the nuclear strength which had been observed by many proton knock-out
%experiments~\cite{VanDerSteenhoven1988547, Lapikas1993297, Kelly:1996hd}.

In the dense and energetic environment of the nucleus, nucleons have a significant probability of interacting at distance near or below 1~fm, even
in light nuclei~\cite{carlson14,lu13}. Protons and neutrons interacting through the strong, short-distance components of the NN potential yield
pairs of nucleons with large momenta, well above the typical scale of the Fermi momentum associated with the shell model picture of nuclei. These
pairs of high-momentum nucleons, the so-called short-range correlations (SRCs), are the dominant source of the high-momentum part of the nuclear
momentum distribution~\cite{SLAC_Measurement_PRC.48.2451, src_john}. Thus, the nuclear momentum distribution has two distinct regions, driven by
very different physics. For momenta below the Fermi momentum, $k_F \approx 300$~MeV/c, we have collective, shell-model behavior which varies
rapidly with A. For momenta above $k_F$, two-body physics dominates and there is a universal structure for all nuclei, driven by the details
of the two-body NN interaction~\cite{RevModPhys.80.189, PhysRevC.53.1689, wiringa14}.

Because the momentum distribution of the nucleus is not an experimental observable, one cannot simply measure this for all nuclei and compare the
high-momentum components. One can, however, test the idea of a universal structure to the high-momentum components by comparing quasi-elastic
scattering from different nuclei at kinematics which require that the struck nucleon have a large initial momentum~\cite{RevModPhys.80.189}.
During the scattering, the electron transfers energy, $\nu$, and momentum $\vec{q}$ to the struck nucleon by exchanging a virtual photon with four
momentum transfer $q^2 = - Q^{2} = \nu^{2}-|\vec{q}|^{2}$. It is useful in this case to define the kinematic variable $x = Q^2/(2M_p\nu)$, where
$M_p$ is the mass of the proton.
%
%This is identical to the Bjorken variable $x$, which correspond to the momentum fraction of the struck quark in deeply-inelastic scattering at
%higher energies, but in the present case, we are looking at coherent scattering from a single nucleon.
%
Elastic scattering from a stationary proton corresponds to $x=1$, while $x>1$ corresponds to high momentum transfer but low energy transfer.
This requires that the virtual photon be absorbed by a proton whose initial momentum is opposite to the momentum of the virtual photon, so that
the large transferred momentum changes the direction of the proton's momentum while minimizing the magnitude of the final momentum and thus the
proton's kinetic energy. Scattering at $x>1$ must involve more than one nucleon as the initial momentum of the struck nucleon must be balanced
by one or more nucleons in the nucleus. The kinematic limit for scattering from a nucleus is $x = M_A/M_p \approx A$, which for scattering
from the deuteron corresponds to $x \approx 2$; scattering at $x>2$ thus involves the participation of at least three nucleons.

Based on these kinematic arguments, one can use $x$ to determine the minimum number of nucleons involved in the interaction. Values of $x$ 
slightly greater than unity requires only a small momentum which can come from either the single-particle contributions or from high-momentum
nucleons associated with SRCs. As $x$ increases, larger momenta are required and for sufficiently large $x$ scattering from nucleons below
the Fermi momentum is forbidden, isolating scattering from SRCs~\cite{RevModPhys.80.189, PhysRevC.53.1689}. Previous measurements at SLAC
and Jefferson Lab revealed a universal form to the high-momentum distributions of the struck nucleons~\cite{SLAC_Measurement_PRC.48.2451,
egiyan2003, PhysRevLett.96.082501, fomin2012, src_john}. In these experiments, the ratio of scattering from a heavy nucleus to the deuteron was
shown to scale, i.e. be independent of $x$ and $Q^2$, for $x \gtorder 1.5$ and $Q^2 \gtorder 1.5$~GeV$^2$, corresponding to scattering from
nucleons with momenta above 300 MeV/c. Other measurements have demonstrated that these high-momentum components are dominated by high-momentum n-p
pairs~\cite{Subedi13062008, korover2014, hen14_science}, meaning that the high-momentum components in all nuclei have a deuteron-like structure.

Taking ratios of heavier nuclei to $^3$He allows a similar examination of the target ratios for $x>2$, where one might expect to see a universal
signature of three-nucleon SRCs - configurations of three nucleons which have large relative but small total momenta. These configurations could
come from either three-nucleon forces or successive hard two-nucleon interactions. The first such measurement~\cite{PhysRevLett.96.082501}
observed ratios which were independent of $x$ above $x=2.25$, and this was taken as an indication that the 3N-SRCs dominated in this region and
extracted the relative contribution of the 3N-SRCs in heavy nuclei compared to $^3$He. However the ratios were measured at relatively low $Q^2$
and the $Q^2$ dependence was not measured. A later experiment~\cite{fomin2012} at higher $Q^2$ yielded $^4$He/$^3$He ratios that were
significantly larger than those from~\cite{PhysRevLett.96.082501}. Thus the question of whether 3N-SRC contributions have been cleanly identified
and observed to dominate at some very large momentum scale is as yet unanswered.

% JRA: Some of this may be better/more concise than bit above.
%
%The first evidence of SRC in inclusive scattering was revealed by the SLAC data~\cite{SLAC_Measurement_PRC.48.2451} with $a_2(A)$ exhibiting a
%plateau between $x\sim1.5$ and $x\sim2$, where 2N-SRC is expected to dominate. A recent measurement from the CLAS data in Hall B at
%JLab~\cite{PhysRevLett.96.082501} also reported the 2N-SRC plateau in the $a_3(A)$ distribution. The latest measurement from the E02-019
%experiment in Hall C at JLab with better precision and a wider range of nuclei, and both $a_2(A)$ and $a_3(A)$ show clear 2N-SRC
%plateau~\cite{fomin2012}. In the $x>2$ region, while the CLAS data claimed a second plateau at $x>2$ in the
%$\sigma_{^{4}He}/\sigma_{^{3}He}$ ratio, E02-019 sees, despite the large error bars, clearly a rise in the $a_3(A)$ distribution. It should be
%noted that both experiments reported data at very different $Q^{2}$ and the kinematical requirement in performing a clean measurement of 3N-SRC is
%not yet well understood.



%The interaction between the virtual photon and the nucleon provides an unique probe to study the nucleon's initial state, e.g., the momentum distribution which is correlated to the scaling function~\cite{West1975263,day_arns, PhysRevC.41.R2474, Boffi19931,RevModPhys.80.189}:
%\begin{equation}
%F(y) = 2\pi\int_{|y|}^{\infty}n(p_{0})\cdot p_{0}dp_{0},
%	\label{fy_mom_eq}
%	\end{equation} 
%	where $n(p_{0})$ is the momentum distribution of the nucleon with the initial momentum $p_{0}$. $y$ is the solution of $M_{A}+\nu = \sqrt{M^{2}+|\vec{q}|^{2}+y^{2}+2y|\vec{q}|}+\sqrt{M_{A-1}^{2}+y^{2}}$ where $M$, $M_{A}$ and $M_{A}$ are the masses of the nucleon, target nucleus A and the (A-1) recoil system, respectively. $F(y)$ can be directly extracted from the experimental QE inclusive cross section:
%	\begin{equation}
%	F(y)=\frac{d^{3}\sigma_{EX}}{dE' d\Omega } \frac{1}{Z\sigma_{p}+N\sigma_{n}} \frac{q}{\sqrt{M^{2}+(y+q)^{2}}},
%	\label{fy_scaling_eq2}
%	\end{equation}
%	where $\sigma_{p}$ and $\sigma_{n}$ are the electron-proton and electron-neutron cross section, respectively.


%Compared with the electron elastic scattering process which is well peaked at $x=Q^{2}/2Mv=1$ (M is the proton mass), the QE process yields a much
%broader peaks at $x=1$ due to the Fermi motion of the nucleon inside the nucleus. By measuring the inclusive QE cross section at $x>1$ with
%$Q^{2}>1~GeV^{2}$, one can carefully map out the SRC in different nuclei by taking the ratio of the cross section per nucleon of the heavy nucleus,
%$A$, to the light nucleus, e.g. deuteron or $\mathrm{^{3}He}$:
%\begin{equation}
%a_{2}(A) = \frac{\sigma_{A}(x,Q^{2})/A}{\sigma_{D}(x,Q^{2})/2},~~~  a_3(A) = \frac{\sigma_{A}(x,Q^{2})/A}{\sigma_{^{3}He}(x,Q^{2})/3},
%\end{equation}



%%{\bf (talk about Doug and Or's Analysis, Phys.Rev.Lett. 114 (2015) 16, 169201, when we discuss our new results)}


