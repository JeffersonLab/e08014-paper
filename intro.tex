% review

Understanding the complex structure of the nucleus remains one of the major uncompleted tasks in nuclear physics, and significant questions remain about the high-momentum components of the nuclear wave-function. This important aspect of nuclear structure is not described by the shell model description. This high-momentum strength appears at low momenta in Mean field calculations~\cite{DeForest1983} which subsequently over predict the cross section for proton knock-out reactions below the Fermi momentum~\cite{VanDerSteenhoven1988547, Lapikas1993297, Kelly:1996hd}.

%This is an important component of nuclear structure that goes beyond the shell model description. Mean field
%calculations~\cite{DeForest1983} do not include these high-momentum components, and so typically overpredict
%the cross section for proton knock-out reactions below the Fermi momentum~\cite{VanDerSteenhoven1988547,
%Lapikas1993297, Kelly:1996hd}.

%Without the inclusion of these high-momentum components, the mean-field calculation using the distorted wave impulse
%approximation~\cite{DeForest1983}  overestimates the nuclear strength which had been observed by many proton knock-out
%experiments~\cite{VanDerSteenhoven1988547, Lapikas1993297, Kelly:1996hd}.

In the dense and energetic environment of the nucleus, nucleons have a significant probability of
interacting at distances $\le$1~fm, even in light nuclei~\cite{carlson14,lu13}. Protons and neutrons
interacting through the strong, short-distance part of the NN interaction give rise to pairs of
nucleons with large momenta. These high-momentum pairs, referred to as short-range correlations (SRCs), are
the primary source of high-momenta in nuclei~\cite{Frankfurt1981215, SLAC_Measurement_PRC.48.2451,
src_john}, well above the typical scale of the Fermi momentum ($k_F \approx 300$~MeV/c) associated with the
shell model picture of nuclear structure. For momenta below $k_F$, we observe shell-model behavior which is
strongly $A$ dependent, while two-body physics dominates above $k_F$ resulting in a universal
structure for all nuclei that is steered by the details of the NN interaction~\cite{RevModPhys.80.189,
PhysRevC.53.1689, wiringa14}.

In the case of inclusive electron scattering it is possible  through kinematics, as follows, to isolate events in which the electron interacts with  high-momentum nucleons. The electron transfers energy, $\nu$, and momentum, $\vec{q}$, to the struck nucleon by exchanging
a virtual photon with four momentum transfer $q^2 = - Q^{2} = \nu^{2}-|\vec{q}|^{2}$. It is useful in this
case to define the kinematic variable $x = Q^2/(2M_p\nu)$, where $M_p$ is the mass of the proton. Elastic
scattering from a stationary proton corresponds to $x=1$, while inelastic scattering must occur at $x<1$. In
a nucleus, the momentum of the nucleon produces a broad quasielastic peak centered near $x=1$. 
%Scattering at
%$x>1$ is beyond the kinematic threshold for scattering from a free nucleon and so must involve more than one
%nucleon. At values of $x$ slightly greater than unity, scattering can occur either from nucleons with the
%modest momenta expected from the mean field, or from high-momentum nucleons associated with SRCs.
Scattering at $x>1$ is beyond the kinematic threshold for scattering from a free nucleon. At values of $x$ slightly greater than unity, scattering can occur either from nucleons with the modest momenta expected from the mean field, or from high-momentum nucleons associated with SRCs. As $x$
increases, larger initial momenta are required until scattering from nucleons below the Fermi
momentum is kinematically forbidden, isolating scattering from high-momentum nucleons associated with
SRCs~\cite{RevModPhys.80.189, PhysRevC.53.1689, src_john, egiyan2003}.

Because the momentum distribution of the nucleus is not a physical observable, one cannot directly extract
and study its high-momentum component. One can, however, test the idea of a universal structure of the
high-momentum components by comparing scattering from different nuclei at kinematics which require that the
struck nucleon have a large initial momentum~\cite{RevModPhys.80.189}. Previous measurements at SLAC and
Jefferson Lab revealed a universal form to the high-momentum distributions of the struck
nucleons~\cite{SLAC_Measurement_PRC.48.2451, egiyan2003, PhysRevLett.96.082501, fomin2012, src_john,
arrington99, arrington01}. In these experiments, the cross section ratios for inclusive scattering from
heavy nuclei to the deuteron were shown to scale, i.e. be independent of $x$ and $Q^2$, for $x \gtorder 1.5$
and $Q^2 \gtorder 1.5$~GeV$^2$, corresponding to scattering from nucleons with momenta above 300 MeV/c.
Other measurements have demonstrated that these high-momentum components are dominated by high-momentum n-p
pairs~\cite{aclander99, tang03, Subedi13062008, korover2014, hen14_science, piasetzky06}, meaning that the high-momentum
components in all nuclei have a deuteron-like structure. While final-state interactions (FSI)
decrease with increasing $Q^2$ in inclusive scattering, FSI between nucleons in the correlated pair may not
disappear. It is typically assumed that the FSI are identical for the deuteron and the deuteron-like pair in
heavier nuclei, and thus cancel in these ratios~\cite{RevModPhys.80.189, src_john}.

This approach can be extended to look for universal behavior arising from 3N-SRCs by examining scattering at
$x>2$ (beyond the kinematic limit for scattering from a deuteron). Within the simple SRC
model~\cite{Frankfurt1981215}, the cross section is composed of scattering from one-body, two-body,
etc... configurations, with the one-body (shell-model) contributions dominating at $x \approx 1$, while
2N-SRCs (3N-SRCs) dominate as $x \to 2 (3)$. Taking ratios of heavier nuclei to $^3$He allows a similar
examination of the target ratios for $x>2$, where the simple SRC model predicts a universal behavior
associated with three-nucleon SRCs (3N-SRCs) - configurations where three nucleons have large relative
momenta but little total momentum. 3N-SRCs could come from either three-nucleon forces or multiple hard
two-nucleon interactions. The first such measurement~\cite{PhysRevLett.96.082501} observed $x$-independent
ratios for $x > 2.25$. This was interpreted as a result of 3N-SRCs dominance in this region.
However the ratios were extracted at relatively small $Q^2$, and the $Q^2$ dependence was not measured. In the experiment of Ref.~\cite{fomin2012}, at higher $Q^2$, the $^4$He/$^3$He ratios  were significantly
larger. Consequently, the question of whether 3N-SRC contributions have been cleanly identified and observed
to dominate at some large momentum scale is as yet unanswered.

% JRA: Some of this may be better/more concise than bit above.
%
%The first evidence of SRC in inclusive scattering was revealed by the SLAC data~\cite{SLAC_Measurement_PRC.48.2451} with $a_2(A)$ exhibiting a
%plateau between $x\sim1.5$ and $x\sim2$, where 2N-SRC is expected to dominate. A recent measurement from the CLAS data in Hall B at
%JLab~\cite{PhysRevLett.96.082501} also reported the 2N-SRC plateau in the $a_3(A)$ distribution. The latest measurement from the E02-019
%experiment in Hall C at JLab with better precision and a wider range of nuclei, and both $a_2(A)$ and $a_3(A)$ show clear 2N-SRC
%plateau~\cite{fomin2012}. In the $x>2$ region, while the CLAS data claimed a second plateau at $x>2$ in the
%$\sigma_{^{4}He}/\sigma_{^{3}He}$ ratio, E02-019 sees, despite the large error bars, clearly a rise in the $a_3(A)$ distribution. It should be
%noted that both experiments reported data at very different $Q^{2}$ and the kinematical requirement in performing a clean measurement of 3N-SRC is
%not yet well understood.


%The interaction between the virtual photon and the nucleon provides an unique probe to study the nucleon's initial state, e.g., the momentum distribution which is correlated to the scaling function~\cite{West1975263,day_arns, PhysRevC.41.R2474, Boffi19931,RevModPhys.80.189}:
%\begin{equation}
%F(y) = 2\pi\int_{|y|}^{\infty}n(p_{0})\cdot p_{0}dp_{0},
%	\label{fy_mom_eq}
%	\end{equation} 
%	where $n(p_{0})$ is the momentum distribution of the nucleon with the initial momentum $p_{0}$. $y$ is the solution of $M_{A}+\nu = \sqrt{M^{2}+|\vec{q}|^{2}+y^{2}+2y|\vec{q}|}+\sqrt{M_{A-1}^{2}+y^{2}}$ where $M$, $M_{A}$ and $M_{A}$ are the masses of the nucleon, target nucleus A and the (A-1) recoil system, respectively. $F(y)$ can be directly extracted from the experimental QE inclusive cross section:
%	\begin{equation}
%	F(y)=\frac{d^{3}\sigma_{EX}}{dE' d\Omega } \frac{1}{Z\sigma_{p}+N\sigma_{n}} \frac{q}{\sqrt{M^{2}+(y+q)^{2}}},
%	\label{fy_scaling_eq2}
%	\end{equation}
%	where $\sigma_{p}$ and $\sigma_{n}$ are the electron-proton and electron-neutron cross section, respectively.


%Compared with the electron elastic scattering process which is well peaked at $x=Q^{2}/2Mv=1$ (M is the proton mass), the QE process yields a much
%broader peaks at $x=1$ due to the Fermi motion of the nucleon inside the nucleus. By measuring the inclusive QE cross section at $x>1$ with
%$Q^{2}>1~GeV^{2}$, one can carefully map out the SRC in different nuclei by taking the ratio of the cross section per nucleon of the heavy nucleus,
%$A$, to the light nucleus, e.g. deuteron or $\mathrm{^{3}He}$:
%\begin{equation}
%a_{2}(A) = \frac{\sigma_{A}(x,Q^{2})/A}{\sigma_{D}(x,Q^{2})/2},~~~  a_3(A) = \frac{\sigma_{A}(x,Q^{2})/A}{\sigma_{^{3}He}(x,Q^{2})/3},
%\end{equation}



%%{\bf (talk about Doug and Or's Analysis, Phys.Rev.Lett. 114 (2015) 16, 169201, when we discuss our new results)}


