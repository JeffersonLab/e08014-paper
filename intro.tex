
% review
Understanding the complex structure of nuclei remains one of the major tasks in nuclear physics. In the dense nuclear medium, constantly moving
nucleons have a significant probability to overlap and therefore create the so-called short-range correlations (SRC). The strong repulsive core
of the nucleon-nucleon (NN) interactions at short distance boosts the correlated nucleons well above the Fermi momentum, while the nucleus remains
in its ground state due to momentum conservation. Without involving these high momentum components, the mean field calculation using the distorted
wave impulse approximation~\cite{DeForest1983} underestimated the nuclear strength which had been observed by many proton knock-out
experiments~\cite{VanDerSteenhoven1988547,Lapikas1993297,Kelly:1996hd}.

Previous data revealed a universal form of the momentum distributions of the struck nucleons~\cite{PhysRevC.53.1689} in the momentum range from
300~MeV to 600~MeV, i.e. the distributions of all nuclei scale to the one of the deuteron. It could be easily understood if the SRC pair (2N-SRC)
shares the similar features among different nuclei. One also suggested that the momentum distributions should scale to the one of $\mathrm{^{3}H}$
or $\mathrm{^{3}He}$ for momentum above 600~MeV, where 3N-SRC configuration should dominate~\cite{src_john}.

Instead of direclty investigating the momentum distribution of the nucleus which is not an experimental observable, one can study the SRC via
inclusive electrons quasi-elastic (QE) scattering off nuclei~\cite{RevModPhys.80.189}. During the scattering, the electron gives up its energy by
emitting a virtual photon with four momentum transfer $q^2 = - Q^{2} = \nu^{2}-|\vec{q}|^{2}$, where $\vec{q}$ and $\nu$ are the momentum and energy
of the virtual photon.

%The interaction between the virtual photon and the nucleon provides an unique probe to study the nucleon's intial state, e.g., the momentum distribution which is correlated to the scaling function~\cite{West1975263,day_arns, PhysRevC.41.R2474, Boffi19931,RevModPhys.80.189}:
%\begin{equation}
%F(y) = 2\pi\int_{|y|}^{\infty}n(p_{0})\cdot p_{0}dp_{0},
%	\label{fy_mom_eq}
%	\end{equation} 
%	where $n(p_{0})$ is the momentum distribution of the nucleon with the initial momentum $p_{0}$. $y$ is the solution of $M_{A}+\nu = \sqrt{M^{2}+|\vec{q}|^{2}+y^{2}+2y|\vec{q}|}+\sqrt{M_{A-1}^{2}+y^{2}}$ where $M$, $M_{A}$ and $M_{A}$ are the masses of the nucleon, target nucleus A and the (A-1) recoil system, respectively. $F(y)$ can be directly extracted from the experimental QE inclusive cross section:
%	\begin{equation}
%	F(y)=\frac{d^{3}\sigma_{EX}}{dE' d\Omega } \frac{1}{Z\sigma_{p}+N\sigma_{n}} \frac{q}{\sqrt{M^{2}+(y+q)^{2}}},
%	\label{fy_scaling_eq2}
%	\end{equation}
%	where $\sigma_{p}$ and $\sigma_{n}$ are the electron-proton and electron-neutron cross section, respectively.


Compared with the electron elastic scattering process which is well peaked at $x=Q^{2}/2Mv=1$ (M is the proton mass), the QE process yields a much
broader peaks at $x=1$ due to the Fermi motion of the nucleon inside the nucleus. By measuring the inclusive QE cross section at $x>1$ with
$Q^{2}>1~GeV^{2}$, one can carefully map out the SRC in different nuclei by taking the cross section ratio of the heavy nucleus, $A$, to the light
nuclei, e.g. deuteron or $\mathrm{^{3}He}$:
\begin{equation}
a_{2}(A) = \frac{2}{A}\frac{\sigma_{A}(x,Q^{2})}{\sigma_{D}(x,Q^{2})},~~~  a_3(A) = \frac{3}{A}\frac{\sigma_{A}(x,Q^{2})}{\sigma_{^{3}He}(x,Q^{2})},
\end{equation}
where $2/A$ or $3/A$ accounts for the possibilities of forming SRC configurations in different nuclei similar to deuteron or $\mathrm{^{3}He}$. 

The first evidence of SRC in inclusive scattering was revealed by the SLAC data~\cite{SLAC_Measurement_PRC.48.2451} with $a_2(A)$ exhibiting a
plateau between $x\sim1.5$ and $x\sim2$, where 2N-SRC is expected to dominate. A recent measurement from the CLAS data in Hall B at
JLab~\cite{PhysRevLett.96.082501} also reported the 2N-SRC plateau in the $a_3(A)$ distribution. The latest measurement from the E02-019
experiment in Hall C at JLab with better precision and a wider range of nuclei, and both $a_2(A)$ and $a_3(A)$ show clear 2N-SRC
plateau~\cite{PhysRevLett.105.212502}. In the $x>2$ region, while the CLAS data claimed a second plateau at $x>2$ in the
$\sigma_{^{4}He}/\sigma_{^{3}He}$ ratio, E02-019 sees, despite the large error bars, clearly a rise in the $a_3(A)$ distribution. It should be
noted that both experiments reported data at very different $Q^{2}$ and the kinematical requirement in performing a clean measurement of 3N-SRC is
not yet well understood.
%{\bf (talk about Doug and Or's Analysis, Phys.Rev.Lett. 114 (2015) 16, 169201, when we discuss our new results)}


